\chapter{绪论}

\section{研究背景与意义}

我们正处在一个信息爆炸的时代,以大数据、人工智能(AI)、物联网(IoT)等为代表的新兴技术蓬勃发展,驱动着社会各领域的深刻变革\cite{lecun2015deep}。
在这些技术的背后,海量数据的处理和分析是核心环节,而矩阵运算,特别是矩阵乘法,作为一种基础且计算密集型的操作,
广泛应用于科学计算、图像处理、信号处理、机器学习、推荐系统、图计算等众多领域\cite{golub2013matrix}。
例如,在深度学习中,神经网络的训练和推理过程涉及大量的卷积和全连接层计算,这些本质上都可以归结为大规模的矩阵或张量运算\cite{chetlur2014cudnn}。
在图计算中,邻接矩阵的乘法可用于发现节点间的路径关系\cite{kepner2011graph}。随着模型复杂度和数据规模的持续增长,对计算能力的需求也呈现出指数级增长的态势,传统的计算模式面临着严峻的挑战。

中央处理器(CPU)作为通用的计算核心,虽然具有强大的逻辑控制能力和灵活性,但在处理高度并行化的数据密集型任务(如大规模矩阵乘法)时,
其固有的串行执行特性和有限的并行处理单元(核心数)往往导致性能瓶颈\cite{hennessy2011computer}。图形处理器(GPU)凭借其众多的计算核心(流处理器)和高内存带宽,
在并行计算领域取得了巨大成功,特别是在稠密矩阵运算和深度学习领域展现出卓越的加速效果\cite{owens2008gpu}。
然而,GPU的体系结构相对固定,对于某些特定类型的计算(例如,具有高度不规则访存模式的稀疏矩阵运算)
或者需要细粒度流水线定制的任务,其效率可能并非最优\cite{bell2009implementing}。此外,GPU通常功耗较高,且其编程模型(如CUDA或OpenCL)虽然强大,但与底层硬件的映射关系不如FPGA直接。

现场可编程门阵列(FPGA)作为一种可编程硬件,提供了一种介于通用处理器和专用集成电路(ASIC)之间的解决方案\cite{xilinx2025ultrascale}。
FPGA内部包含大量的可配置逻辑块(CLB)、存储单元(BRAM)、DSP单元等,用户可以通过硬件描述语言(HDL)或高层次综合(HLS)工具对其进行编程,
实现高度定制化的硬件加速器\cite{nane2015survey}。相比CPU,FPGA能够实现真正在硬件层面的并行和流水线处理,大幅提升计算密集型任务的性能功耗比。
相比GPU,FPGA的架构灵活性允许针对特定算法进行深度优化,例如为稀疏矩阵的不同存储格式(如CSR, CSC, COO等)设计专门的处理单元和访存逻辑\cite{zhuo2005sparse}。
相比ASIC,FPGA虽然在绝对性能和功耗上可能稍逊一筹,但其可重复编程的特性大大降低了设计成本和风险,缩短了开发周期,尤其适合算法快速迭代或需要适应多种应用场景的领域。

然而,传统的基于FPGA的加速器设计通常是“静态”的,即一旦配置了FPGA,其硬件逻辑就固定下来,直到下一次完全重新配置。
这种模式对于功能固定的应用是有效的,但在许多现代应用场景中,计算任务的需求可能是动态变化的。
例如,一个复杂的数据处理流程可能包含多个阶段,每个阶段需要不同的加速核心;
或者,系统需要处理不同格式、不同稀疏度的稀疏矩阵,为每种情况设计一个最优的静态加速器并在需要时切换,
会导致整个FPGA的重配置,这个过程通常耗时较长(从毫秒级到秒级),中断服务,对于需要快速响应或持续服务的系统是不可接受的\cite{blodget2003self}。

为了克服静态FPGA设计的局限性,动态部分重构(Dynamic Partial Reconfiguration, DPR)技术应运而生\cite{xilinx2024vivado,vipin2018fpga}。
DPR允许在FPGA运行时,仅对其内部指定的一部分区域(称为可重构分区,Reconfigurable Partition, RP)进行重新编程,
加载新的硬件逻辑(称为可重构模块,Reconfigurable Module, RM),而FPGA的其他部分(静态区域和其他RP)可以保持正常工作,不被中断\cite{beckhoff2012go,koch2012partial}。
这项技术极大地增强了FPGA的灵活性和资源利用率。通过DPR,可以在有限的FPGA资源上分时复用不同的加速功能,或者根据输入数据的特性
(如稀疏矩阵的格式或密度)动态加载最优的处理模块,从而实现“按需定制”的硬件加速。

将DPR技术应用于矩阵乘法加速,特别是涉及到稀疏矩阵的复杂场景,具有重要的研究价值和应用前景。
稀疏矩阵在现实世界中无处不在(如社交网络关系、物理模拟、自然语言处理中的词袋模型等),
其存储和计算方式与稠密矩阵截然不同,且存在多种存储格式,不同格式适用于不同的计算场景和优化策略\cite{saad2003iterative}。
一个能够动态切换稀疏矩阵处理逻辑(如不同格式的解压缩、压缩、乘法核心)的FPGA加速系统,将能更高效地适应多样化的应用需求,
提高硬件资源的利用率和系统的整体性能。

本设计正是基于上述背景,旨在探索和实现一个基于FPGA动态部分重构技术的矩阵乘法加速系统。
系统以Xilinx Kria KV260 Vision AI Starter Kit为硬件平台,该平台搭载了Zynq UltraScale+ MPSoC,
集成了强大的ARM处理核心和FPGA可编程逻辑,构成了典型的异构计算系统\cite{xilinx2024kv260}。
利用MPSoC的FPGA部分实现硬件加速,利用ARM核心运行Linux操作系统进行任务调度、资源管理和与FPGA的交互。
通过在FPGA上划分多个可重构分区,并利用高层次综合(HLS)语言(如C/C++)设计可重构的矩阵运算模块
(包括稀疏矩阵解压、稠密矩阵乘法、稀疏矩阵压缩等),实现了这些模块在运行时的动态加载和切换。
系统通过标准的AXI接口协议实现FPGA内部模块间以及FPGA与处理器系统(PS)端内存的数据交互。
结合Xilinx Runtime (XRT) 库,在运行于CPU上的应用程序可以方便地控制FPGA上的可重构模块,实现异构协同计算。

\section{国内外研究现状}

\subsection{静态FPGA加速}

许多研究集中于优化特定类型的矩阵乘法。对于稠密矩阵乘法,研究者们探索了各种并行计算架构(如脉动阵列)、
分块策略、片上存储优化以及利用HLS简化设计流程的方法,以期达到接近理论峰值的性能\cite{de2020flexible,pucscacsu2024systolic}。
对于稀疏矩阵乘法(SpMV, SpMM),挑战主要在于处理不规则的内存访问和计算模式。
研究工作通常针对特定的稀疏格式设计专门的硬件结构,
利用流水线、数据预取、负载均衡等技术来缓解访存瓶颈和提高计算单元利用率\cite{dorrance2014scalable}。

\subsection{动态重构技术应用}

DPR技术本身已经相对成熟,并在通信、自适应滤波、软件定义无线电等领域有所应用\cite{bobda2007introduction}。
在计算加速领域,DPR被用于根据需要加载不同的加密算法、图像处理算子或数据压缩算法\cite{gonzalez2003using,ram2020dynamic}。
将DPR用于矩阵运算的研究相对较少。可行的初步探索包括根据矩阵的尺寸或特性动态调整FPGA上矩阵乘法器的并行度或数据位宽,
或利用DPR在不同类型的数值计算核心(如浮点加法器、乘法器)之间切换。

\subsection{异构计算与运行时管理}

随着SoC FPGA(如Xilinx Zynq系列)的普及,基于FPGA的异构计算系统成为主流。
研究者们关注如何高效地在处理器和FPGA之间划分任务、传输数据以及管理FPGA上的硬件资源\cite{mitola2002software}。
Xilinx PYNQ框架和XRT运行时库等工具的出现,极大地简化了在高级操作系统(如Linux)层面调用FPGA加速器的开发流程。

\subsection{现有研究的局限性}

尽管已有大量关于FPGA矩阵加速和DPR技术的研究,但将两者深度结合,特别是针对包含多种稀疏/稠密矩阵操作、
需要动态适应不同稀疏格式的复杂计算流,进行系统性设计和实现的研究尚不多见。
现有工作要么侧重于静态优化单一类型的矩阵运算,要么DPR的应用场景相对简单,未能充分发挥DPR在处理复杂、
多变矩阵计算任务流中的潜力。此外,如何在HLS设计流程中高效地集成DPR,
以及如何通过XRT等运行时环境有效管理多个可重构分区的动态加载和任务调度,也是需要进一步探索的问题。

\section{本设计的主要工作与创新点}

本毕业设计针对现有研究的不足,着眼于设计并实现一个动态可重构的FPGA矩阵乘法加速系统,旨在提供一个既高效又灵活的解决方案,
以适应现代应用中复杂多变的矩阵计算需求。具体工作和创新点如下:
\begin{description}
\item[基于DPR的灵活矩阵运算硬件架构设计] 在Xilinx KV260平台的FPGA上,
  设计并实现了包含三个独立可重构分区(RP)的硬件架构。
  这种多RP架构允许多个不同的矩阵运算模块(RM)共存或被快速替换,为实现复杂的计算流水线或并行处理不同任务提供了硬件基础。
\item[面向多类型矩阵运算的可重构模块开发] 使用Vitis HLS工具,以C/C++语言开发了一系列针对不同矩阵运算任务的RM,包括:
\begin{itemize}
  \item 稀疏矩阵解压(将特定稀疏格式转换为稠密格式)
  \item 稠密矩阵乘法
  \item 稀疏矩阵压缩(将稠密格式转换为特定稀疏格式)
  \item (隐含支持)稀疏矩阵格式转换(可通过解压+压缩RM组合实现)
  \item (隐含支持)稀疏矩阵乘法(可通过解压+稠密乘法,或解压+稠密乘法+压缩等RM组合/序列实现)
\end{itemize}
这些模块被设计为可动态加载到RP中,使得系统能够根据具体任务需求,配置相应的硬件加速逻辑,并切换对不同稀疏矩阵格式的处理。
\item[标准化接口与异构系统集成] 设计的RM均采用标准的AXI接口协议。
AXIMM接口用于高效访问KV260的共享DDR内存,实现大规模数据的吞吐;
AXIS接口用于连接不同的RM,支持在FPGA内部构建数据流驱动的计算流水线。
系统运行在KV260的ARM处理器上的Ubuntu Linux操作系统,通过Xilinx Runtime (XRT)库,
实现了上层软件对FPGA硬件资源(包括DPR操作和RM任务执行)的统一管理和调用,构建了一个完整的异构加速计算平台。
\end{description}

本设计的创新之处在于:系统性地将动态部分重构技术应用于涵盖多种(稀疏、稠密、转换)矩阵运算的FPGA加速场景,
并构建了一个包含多RP、HLS设计的RM、标准AXI接口、以及基于Linux+XRT的软硬件协同控制的完整异构系统。
这为应对未来更复杂、更动态的计算挑战提供了一种有潜力的技术途径。

\section{论文结构安排}

本论文共分为七章,结构安排如下:
\textbf{第一章:绪论}主要介绍研究背景、意义、国内外研究现状、本设计的主要工作与创新点以及论文的结构安排。
\textbf{第二章:相关技术概述}详细介绍本设计所涉及的关键技术,包括矩阵运算基础、FPGA技术原理、异构计算系统。
\textbf{第三章:系统总体设计}阐述动态可重构矩阵乘法加速系统的整体架构,包括硬件系统设计
(FPGA分区规划、静态与动态区域划分、接口设计)和软件系统设计(操作系统层面、XRT应用层面、任务调度逻辑)。
\textbf{第四章:可重构模块硬件实现}详细介绍使用Vitis HLS设计和实现各个可重构矩阵运算模块(稀疏解压、稠密乘法、稀疏压缩等)的过程,
包括算法分析、HLS优化(并行、流水线)、接口实现。
\textbf{第五章:软件系统实现与集成}描述在KV260的ARM处理器上配置Linux环境、
编写主机端应用程序以控制DPR流程(加载/卸载RM)、管理数据传输以及调用FPGA执行加速任务的具体实现方法。
\textbf{第六章:实验结果与分析}对实现的系统进行测试和评估。包括功能验证、
稠密矩阵乘法在FPGA上的加速性能测试(与纯CPU实现对比),以及系统灵活性和资源利用率的分析。
\textbf{第七章:总结与展望}总结本论文完成的主要工作和取得的研究成果,分析存在的不足,并对未来可以进一步研究的方向进行展望。
