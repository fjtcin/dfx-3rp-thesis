\chapter{相关技术概述}

本章旨在详细介绍本工作所涉及的核心技术。
这些技术涵盖了从基础算法理论到硬件平台、设计方法学、接口协议以及软件运行时环境等多个层面。
深入理解这些技术对于后续章节中系统设计、实现与评估的阐述至关重要。
本章将依次介绍矩阵运算基础、FPGA技术原理、高层次综合(HLS)、动态部分重构(DPR/DFX)技术、
AXI总线协议、异构计算系统、Xilinx Kria KV260平台特性以及Xilinx Runtime(XRT)环境。

\section{矩阵运算基础}

矩阵作为一种重要的数据结构,在科学计算、图像处理、机器学习、组合优化等诸多领域扮演着核心角色。
矩阵运算,尤其是矩阵乘法,通常是这些应用中计算量最为密集的部分之一。
根据矩阵中非零元素的分布,矩阵可分为稠密矩阵和稀疏矩阵。稠密矩阵的元素大多为非零值,
而稀疏矩阵则包含大量的零元素。针对稀疏矩阵的特性,发展出了多种压缩存储格式,
如坐标列表(COO)、压缩稀疏行(CSR)、压缩稀疏列(CSC)等,以节省存储空间并优化计算效率。
本项目涉及的核心计算任务包括稀疏矩阵与稠密矩阵之间的相互转换(解压与压缩)、
稀疏矩阵格式之间的转换,以及稠密矩阵乘法和稀疏矩阵乘法。这些运算,特别是涉及大规模矩阵时,对计算性能提出了极高要求,从而驱动了对硬件加速方案的探索。

\section{FPGA技术原理}

现场可编程门阵列(FPGA)是一种半定制电路,其硬件结构可以在制造完成后由用户根据需求进行配置。
FPGA内部主要由可配置逻辑块(CLB)、输入输出块(IOB)、
可编程布线资源以及嵌入式存储器(如BRAM)、数字信号处理单元(DSP Slice)等构成。
用户通过加载特定的配置文件(比特流)来定义这些单元的功能及其互连方式,从而实现所需的数字逻辑电路。
与通用处理器相比,FPGA能够实现高度的并行计算和深流水线操作,充分利用硬件资源,
从而在特定计算密集型任务上展现出显著的性能优势和能效比。其可重构性也为算法的迭代升级和功能调整提供了灵活性。

\section{高层次综合}

高层次综合(High-Level Synthesis, HLS)是一种设计方法,
它允许开发者使用诸如C、C++或SystemC等高层次语言来描述硬件行为,
然后通过HLS工具将其自动转换为低层次的硬件描述语言(HDL),如Verilog或VHDL。
HLS的出现极大地提高了FPGA的设计效率,缩短了开发周期,并使得不熟悉底层HDL的软件工程师也能够参与到FPGA开发中。
在本项目中,可重构模块采用Vitis HLS进行编写,通过利用HLS提供的并行化、流水线化等优化指令(pragmas),
能够有效地将算法映射到FPGA的硬件资源上,实现高效的硬件加速器设计。

\section{动态部分重构技术}

动态部分重构(Dynamic Partial Reconfiguration, DPR),在Xilinx平台中也称为Dynamic Function eXchange (DFX),是一项先进的FPGA技术。
它允许在FPGA正常运行期间,对其内部的特定区域(即可重构分区, Reconfigurable Partition, RP)进行动态地、部分地重新编程,
而FPGA的其他部分保持原有功能并继续工作。这项技术为系统带来了极大的灵活性,
使得FPGA能够根据应用需求实时切换不同的硬件加速模块(可重构模块, Reconfigurable Module, RM),
或者在线更新硬件功能,而无需中断整个系统的运行。在本项目中,通过在FPGA上成功布局3个可重构分区,
并部署不同的矩阵运算模块,实现了根据计算任务动态加载和切换功能,例如在不同稀疏矩阵格式处理模块间进行切换,以适应多样化的加速场景。

\section{AXI总线协议}

高级可扩展接口(Advanced eXtensible Interface, AXI)是ARM公司提出的一种高性能、
高带宽、低延迟的片上总线协议,已成为业界标准,广泛应用于SoC(System-on-Chip)设计中。
AXI协议定义了主设备(Master)和从设备(Slave)之间的通信规范,支持多种通信模式。
本项目中主要使用了两种AXI接口:AXI Memory Mapped (AXIMM) 协议和AXI Stream (AXIS) 协议。
AXIMM协议用于可重构分区对KV260的内存进行直接访问(DMA),实现高效的数据读写;
AXIS协议则用于可重构模块之间的互联,支持高速、连续的数据流传输,非常适合流水线式的数据处理架构。

\section{异构计算系统}

异构计算系统是指在一个系统中集成多种不同类型计算单元(如CPU、GPU、FPGA、DSP等)的计算平台。
这种架构旨在结合不同处理单元的优势,例如CPU擅长处理复杂的控制流和非结构化任务,而FPGA则擅长执行大规模并行计算和定制化的数据处理。
通过合理的任务划分和协同工作,异构计算系统能够实现比单一类型处理器更高的性能和能效。
本项目基于Xilinx KV260 MPSoC构建了一个异构计算系统,其中ARM Cortex-A53 CPU负责运行操作系统(Ubuntu Linux)和上层应用程序,
并调度FPGA上的硬件加速模块执行计算密集型的矩阵运算任务。

\section{Xilinx Kria KV260平台特性}

Xilinx Kria KV260视觉AI入门套件是基于Kria K26系统级模块(SOM)的开发平台。
K26 SOM是一款多处理器系统芯片(MPSoC),集成了四核ARM Cortex-A53处理系统(PS)和拥有丰富可编程逻辑(PL)资源的FPGA。
KV260平台专为边缘AI和视觉应用设计,但其强大的异构处理能力和灵活的接口使其也适用于其他加速计算任务。
它提供了DDR4内存、多种外设接口以及对Xilinx开发工具链的良好支持。其MPSoC架构天然支持PS与PL之间的高效协同,
是实现本项目中CPU控制、FPGA加速的理想硬件基础。FPGA部分支持动态部分重构,为实现灵活可变的加速器提供了硬件保障。

\section{Xilinx运行时环境}

Xilinx Runtime (XRT) 是一个开源的、标准化的软件平台,旨在简化主机CPU与Xilinx FPGA加速器之间的交互。
XRT包括用户空间库、API以及内核驱动程序,它为应用程序提供了一个统一的接口来管理和控制FPGA上的加速内核,
无论这些内核是部署在Alveo数据中心加速卡还是像KV260这样的嵌入式SoC平台上。XRT负责处理诸如加载比特流(包括部分重构的比特流)、
分配和迁移内存缓冲区、调度内核执行以及收集性能数据等任务。在本项目中,部署在KV260的CPU上的Ubuntu Linux操作系统中,
应用程序通过XRT提供的API来调用和管理FPGA上的动态可重构矩阵运算模块,实现了高效的异构加速计算流程。
