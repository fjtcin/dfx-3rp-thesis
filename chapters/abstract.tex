% !TeX root = ../main.tex

\ustcsetup{
  keywords  = {可重构计算, FPGA加速, 稀疏矩阵},
  keywords* = {Reconfigurable Computing, FPGA Acceleration, Sparse Matrix},
}

\begin{abstract}
本工作在Xilinx KV260多处理器系统芯片的FPGA上,成功实现了3个可重构分区的布局,并在上面分别部署了稀疏矩阵解压、稠密矩阵乘法、稀疏矩阵压缩的可重构模块。
可重构分区通过AXIMM协议对KV260的内存直接访问,同时通过AXIS协议进行互联。
可重构模块使用高层次综合语言编写,能充分利用FPGA的硬件资源,实现并行、流水线地处理。
稠密矩阵乘法使用分块与脉动阵列优化,每个块大小12\texttimes{}12,对于128\texttimes{}128矩阵乘法相较CPU朴素算法实现了29\texttimes{}的加速比。

在KV260的CPU上,编写程序调用FPGA上的计算模块,实现了异构加速计算。
支持COO, CSR和CSC三种稀疏格式的稀疏矩阵解压/压缩/乘法操作。所有贡献开源于\url{github.com/fjtcin/dfx-3rp}。
\end{abstract}

\begin{abstract*}
This work successfully implements a layout of three reconfigurable partitions on the FPGA of a Xilinx KV260 multi-processor system-on-chip (MPSoC).
Reconfigurable modules for sparse matrix decompression, dense matrix multiplication, and sparse matrix compression are deployed on these partitions respectively.
The reconfigurable partitions directly access the KV260's memory via the AXI Memory Mapped protocol and are interconnected using the AXI Stream protocol.
The reconfigurable modules are written in the high-level synthesis language,
enabling full utilization of FPGA hardware resources for parallel and pipelined processing.
The dense matrix multiplication is optimized using tiling and a systolic array, with each block sized at 12\texttimes{}12.
For a 128\texttimes{}128 matrix multiplication, this achieves a 29× speedup compared to a naive CPU algorithm.

Applications have been developed to invoke the computational modules on the FPGA,
realizing heterogeneous accelerated computing. The system supports sparse matrix decompression,
compression, and multiplication operations for COO, CSR, and CSC sparse formats. All contributions are open-source at \url{github.com/fjtcin/dfx-3rp}.
\end{abstract*}
