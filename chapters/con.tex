\chapter{总结}

\section{本论文主要工作总结}

本论文围绕在Xilinx Kria KV260多处理器系统芯片(MPSoC)上构建一个高性能、高灵活性的动态可重构矩阵乘法加速系统展开研究与实现。
面对传统固定功能加速器在处理多样化矩阵运算任务及适应不同数据格式方面的局限性,本文提出并成功实现了一个基于动态部分重构(DFX)技术的异构计算解决方案。

首先,在系统总体设计层面(第三章),我们规划了基于KV260的软硬件协同架构。
硬件上,在FPGA的可编程逻辑(PL)部分划分了静态区域和三个独立的可重构分区(RP)。
静态区域负责基础平台支持,而RP则用于动态加载不同的计算模块。
软件上,在处理器系统(PS)端的ARM处理器上部署了Ubuntu Linux操作系统,并利用Xilinx Runtime (XRT) 作为主机应用程序与FPGA硬件交互的桥梁。

其次,在可重构模块硬件实现层面(第四章),我们采用Vitis高层次综合(HLS)语言,设计并优化了一系列核心矩阵运算的可重构模块(RM)。
这些模块包括稀疏矩阵解压模块、采用脉动阵列结合分块策略的稠密矩阵乘法模块以及稀疏矩阵压缩模块。
通过精心设计的并行与流水线优化,这些RM能够高效利用FPGA资源。模块间通过AXI4 Stream (AXIS) 协议实现高速数据互联,
并通过AXI4 Memory Mapped (AXIMM) 协议直接访问KV260的系统内存。

再次,在软件系统实现与集成层面(第五章),我们阐述了在KV260的ARM处理器上配置Linux环境,
介绍了主机端C++应用程序的设计,实现异构计算。

最后,通过一系列实验(第六章),我们对所构建的系统进行了全面的功能验证、
稠密矩阵运算任务的加速性能评估,并分析了系统的灵活性和硬件资源利用率。

\section{存在的不足与挑战}

尽管本系统取得了一定的成果,但在研究和实现过程中也清晰地揭示了一些固有的不足之处以及面临的技术挑战。
一个核心的考量在于动态部分重构(DPR)本身引入的开销。具体而言,加载部分比特流所需的时间,在实验中表现为5毫秒左右,
对于那些执行时间极短或者需要极其频繁切换可重构模块(RM)的计算任务而言,这一延迟可能显著影响甚至抵消FPGA的加速效益。
与此同时,数据传输瓶颈也是限制整体性能提升的关键因素。尽管采用了AXIMM协议实现直接内存访问,PS与PL之间的数据交互带宽和延迟,
以及在某些复杂流水线中RM之间可能需要通过DDR进行的中间数据交换,依然对加速比的上限构成了制约,特别是对于那些计算密度不高、数据吞吐量大的模块。

此外,Kria KV260作为一款入门级的MPSoC平台,其FPGA内部的逻辑资源、DSP单元以及BRAM总量是相对有限的。
这一硬件平台的资源约束,自然地限制了单个可重构模块所能达到的最大复杂度,例如稠密矩阵乘法中脉动阵列的规模,
同时也可能制约了可同时部署的RP数量或每个RP的大小,从而影响了系统整体的并行处理能力。设计和调试DFX系统本身也带来了额外的复杂性。
相较于传统的静态FPGA设计流程,DFX要求更精细的分区规划、严格的接口隔离规则,以及更为繁琐的部分比特流生成、管理与验证过程,
这些都对开发效率和调试难度提出了更高的要求。最后,从应用范围来看,当前实现的RM库虽然覆盖了基础的核心矩阵运算,
但其在支持更多高级矩阵算法或更广泛、更特殊的稀疏数据格式方面,其完备性仍有提升空间,这可能限制了系统在某些特定专业领域的直接适用性。

\section{未来工作展望}

基于当前工作的成果和已识别的不足,未来的研究可以向多个富有前景的方向拓展和深化。
首要的努力方向应聚焦于进一步降低DPR开销并提升系统的动态响应能力。
这可能涉及对更先进重构技术的探索,例如研究配置数据压缩方法、利用未来Xilinx工具链可能支持的更细粒度重构机制,
或者探索硬件加速的配置管理单元设计,乃至结合预测性加载和智能化调度策略,以期在用户层面隐藏或显著减少DPR切换带来的延迟。
与此同时,持续优化数据通路和内存访问效率也至关重要。这包括深入研究片上高速缓存(如BRAM/URAM)的更高效管理和分配策略,
探索更先进的DMA传输模式以减少CPU干预和传输延迟,并强化RM之间通过AXIS等方式的直接流式数据传输,最大限度地避免数据经由外部DDR的低效中转。

在提升系统功能性和易用性方面,可以致力于扩展可重构模块(RM)的功能覆盖范围和设计灵活性。
这包括开发一个更加丰富和多样化的RM库,以支持如矩阵分解、迭代求解器、特征值计算等更高级的矩阵运算,并兼容更多标准或用户定义的稀疏数据格式。
同时,研究高度参数化的RM设计方法,使得RM能够在一定范围内通过寄存器配置等轻量级方式调整其内部并行度、数据位宽或特定算法路径,
从而在完全DPR和静态配置之间找到更优的平衡点,提升模块的复用性和适应性。
软件层面,开发更高层次的抽象接口或领域特定语言(DSL)将极大简化用户对复杂DFX系统的编程和任务部署,
未来的系统甚至可以集成基于机器学习的智能调度器,使其能够根据实时工作负载特性和系统状态自动优化RM的选择、加载时序及RP资源分配。

最后,将本研究中积累的设计理念和实践经验应用于新兴的FPGA架构和异构计算平台,无疑是一个充满机遇的研究方向。
例如,Xilinx Versal ACAP等新一代器件集成了更为强大的AI引擎、智能引擎以及更成熟和灵活的DFX支持能力,
在这些新平台上重构和扩展本系统,有望实现性能和能效的飞跃。此外,对系统在不同工作负载和RM配置下的功耗进行细致的建模、分析与优化,
特别是针对边缘计算等功耗敏感应用场景,开发面向低功耗的DFX策略和RM设计方法,也将是未来工作中一个具有实际应用价值的研究点。
