\chapter{总结}

本论文围绕在Xilinx Kria KV260多处理器系统芯片(MPSoC)上构建一个高性能、高灵活性的动态可重构矩阵乘法加速系统展开研究与实现。
面对传统固定功能加速器在处理多样化矩阵运算任务及适应不同数据格式方面的局限性,本文提出并成功实现了一个基于动态部分重构(DFX)技术的异构计算解决方案。

首先,在系统总体设计层面(第三章),我们规划了基于KV260的软硬件协同架构。
硬件上,在FPGA的可编程逻辑(PL)部分划分了静态区域和三个独立的可重构分区(RPs)。
静态区域负责基础平台支持,而RPs则用于动态加载不同的计算模块。
软件上,在处理器系统(PS)端的ARM处理器上部署了Ubuntu Linux操作系统,并利用Xilinx Runtime (XRT) 作为主机应用程序与FPGA硬件交互的桥梁。

其次,在可重构模块硬件实现层面(第四章),我们采用Vitis高层次综合(HLS)语言,设计并优化了一系列核心矩阵运算的可重构模块(RMs)。
这些模块包括稀疏矩阵解压模块、采用脉动阵列结合分块策略的稠密矩阵乘法模块以及稀疏矩阵压缩模块。
通过精心设计的并行与流水线优化,这些RM能够高效利用FPGA资源。模块间通过AXI4-Stream (AXIS) 协议实现高速数据互联,
并通过AXI4 Memory Mapped (AXIMM) 协议直接访问KV260的系统内存。

再次,在软件系统实现与集成层面(第五章),我们详细阐述了在KV260的ARM处理器上配置Linux环境、安装XRT,
并重点介绍了主机端C++应用程序的设计。该应用程序负责解析用户请求,通过XRT动态管理FPGA上的RPs(加载/卸载RMs),
高效组织数据在PS与PL间的传输,并精确控制FPGA加速内核的执行流程,实现了对整个异构加速系统的有效调度。

最后,通过一系列实验(第六章),我们对所构建的系统进行了全面的功能验证、DPR操作性能测量、
核心矩阵运算任务的加速性能评估(与CPU基准对比),并分析了系统的灵活性和硬件资源利用率。
