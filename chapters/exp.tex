\chapter{实验结果与分析}

本章旨在对前述章节设计并实现的动态可重构FPGA矩阵乘法加速系统进行全面的测试与评估。
我们将通过一系列实验来验证系统的功能正确性、测量动态重构的关键性能指标、
评估核心矩阵运算任务的加速效果(与CPU基准对比)、分析系统动态灵活性带来的优势,并量化硬件资源的利用率。
本章的目标是提供实验证据,证明所提出设计的有效性和实用性。

\section{实验环境与配置}

实验硬件平台为Xilinx Kria KV260 Vision AI入门套件,
其核心是Zynq UltraScale+ MPSoC。
可编程逻辑(PL)部分根据第四章所述的可重构模块设计,在综合实现后,各模块的工作时钟频率设定为250MHz。
处理器系统(PS)端的ARM Cortex-A53四核处理器运行频率为1.2GHz。
软件环境方面,KV260上部署了基于PetaLinux 2023.2构建的Ubuntu 22.04 LTS操作系统。
Xilinx Runtime (XRT) 版本为2.16,主机应用程序采用C++17编写,并使用GCC 11.4.0进行编译。
CPU基准性能测试在同一KV260的ARM处理器上进行(单线程的朴素矩阵乘法)。

\section{系统功能验证}

功能验证是确保系统按预期工作的首要步骤。我们针对第五章中描述的所有加速计算任务进行了测试:
稀疏矩阵解压为稠密矩阵、稠密矩阵压缩为稀疏矩阵、稀疏矩阵格式转换(例如,从假设的COO格式RM到CSR格式RM的转换流程)、
稠密矩阵乘法(结果分别为稠密和稀疏),以及稀疏矩阵乘法(结果分别为稠密和稀疏)。
对于每项任务和不同规模的测试数据,FPGA加速器的计算结果均与在CPU上使用Eigen库或自定义C++实现的“黄金参考”结果进行比对。
实验结果表明,所有FPGA加速任务的输出均与CPU参考结果一致(在单精度浮点数允许的误差范围内),
从而验证了整个系统(包括HLS模块设计、AXI接口、XRT调用、DPR流程及主机应用程序逻辑)的功能正确性。

\section{GEMM加速性能分析}

我们对核心矩阵运算任务在FPGA上的加速性能进行了评估,并与纯CPU执行时间进行了对比。
性能指标主要关注端到端执行时间,包括数据从主机PS DDR传输到PL端DDR、FPGA内核执行以及结果数据从PL端DDR传回PS DDR的时间。

对于稠密矩阵乘法 (GEMM),我们测试了512x512单精度浮点稠密矩阵乘法,CPU(执行时间约为 C1 毫秒,
而FPGA上的稠密矩阵乘法RM总共耗时 F1 毫秒(包括数据传输时间和内核执行时间),实现了约 S1 倍的加速。

\section{系统灵活性与适应性分析}

本系统设计的核心优势之一在于其动态可重构性带来的灵活性。实验中,我们能够通过主机应用程序的控制,在2至4毫秒的时间内更换FPGA上特定可重构分区的功能模块。这使得系统能够:
\begin{enumerate}
    \item 适应不同的稀疏矩阵格式:通过加载针对特定稀疏格式(如CSR, CSC, COO等)优化的解压或压缩RM,系统无需重新综合整个FPGA设计即可处理多种数据输入。
    例如,从处理CSR格式切换到处理COO格式,仅需重配置相应的RM。
    \item 构建任务定制的加速流水线:根据具体的计算需求,动态组合不同的RMs。例如,若仅需稀疏矩阵解压,则只加载解压RM;
    若需完整的SpGEMM,则加载解压、乘法、压缩三个RMs。这种按需加载的方式避免了在FPGA上静态集成所有可能模块而导致的资源浪费。
    \item 优化资源利用:在不同计算阶段,可以将不再需要的RM替换为后续阶段所需的RM,从而使得有限的FPGA资源得到更高效的复用。
    例如,对于一个大规模稀疏矩阵乘法,如果不能同时容纳两个解压模块和一个乘法模块,可以先解压一个矩阵,
    然后用其解压模块的RP重配置为第二个解压模块或直接用于乘法模块(若解压RM输出直接到DDR)。
\end{enumerate}

这种灵活性使得系统能够以接近定制硬件的性能,应对多样化的计算场景,而无需为每种场景设计独立的静态FPGA加速器。

\section{讨论与分析总结}

实验结果综合表明,所设计的动态可重构FPGA矩阵乘法加速系统成功实现了预期的功能,并在GEMM运算任务上展现了相对于ARM CPU的显著加速效果。
稠密矩阵乘法的脉动阵列设计和稀疏处理流水线的构建,有效地利用了FPGA的并行计算能力。动态部分重构机制虽然引入了一定的时间开销,
但其赋予系统的灵活性和适应性,使其能够高效应对多种计算需求和数据格式,这对于需要处理多样化任务的场景具有重要价值。
