\chapter{系统总体设计}

在深入探讨具体的硬件实现和软件编程细节之前,本章将阐述动态可重构矩阵乘法加速系统的整体架构。
系统设计遵循软硬件协同设计(Hardware/Software Co-design)的理念,旨在充分利用Xilinx Kria KV260平台的异构计算能力,
结合动态部分重构(Dynamic Partial Reconfiguration, DFX)技术,实现一个既高性能又具备高度灵活性的矩阵运算加速解决方案。
本章将分别从硬件系统架构和软件系统架构两个层面进行阐述,并明确两者之间的交互方式与核心接口设计,为后续章节的详细实现奠定基础。

\section{硬件系统架构}

硬件系统的核心是Xilinx Kria KV260多处理器系统芯片(MPSoC),
其集成了可编程逻辑(Programmable Logic, PL)即FPGA,以及处理系统(Processing System, PS)即ARM CPU。
本设计的FPGA部分充分利用了其并行处理能力和动态可重构特性。

在FPGA的整体规划上,我们将其划分为静态区域(Static Region)和动态区域(Dynamic Regions),
即所谓的可重构分区(Reconfigurable Partitions, RPs)。静态区域承载了系统运行所必需的基础逻辑,
例如与PS的接口控制器、时钟管理、中断管理以及用于加载动态模块的重构控制器(如ICAP)。本设计成功实现了三个独立的可重构分区,
这三个分区为动态加载不同的计算模块提供了物理基础。每个可重构分区均设计为能够容纳一个可重构模块(Reconfigurable Module, RM)。
具体部署的RMs包括稀疏矩阵解压模块、采用脉动阵列结合分块策略的稠密矩阵乘法模块以及稀疏矩阵压缩模块。
这些模块均采用Vitis高层次综合(HLS)语言编写,通过精心设计的pragma指令,实现了高度的并行计算和流水线操作,从而最大化地利用FPGA的硬件资源,提升运算效率。

接口设计是确保数据高效流转和模块协同工作的关键。在可重构分区与KV260片上内存的交互方面,
采用了AXI Memory Mapped (AXIMM) 主接口协议。这使得每个动态加载的RM都能够直接、
高效地对系统主存进行读写操作,为大规模矩阵数据的传输提供了高带宽通道。
同时,为了实现可重构模块之间的数据流传递和协同处理,例如将解压后的数据直接送入乘法模块,
我们设计了基于AXI Stream (AXIS) 协议的互联接口。这种流式接口非常适合于流水线式的处理流程,
能够有效减少数据在模块间传递的延迟。静态区域与动态区域之间,以及动态区域与PS之间的控制与状态信号交互,则通过标准的AXI Lite接口或GPIO实现。

\section{软件系统架构}

软件系统架构是建立在KV260的ARM处理器之上,负责管理硬件资源、调度计算任务以及提供用户交互接口。
整个软件栈可以划分为操作系统层面、Xilinx运行时(XRT)应用层面以及顶层的任务调度逻辑。

在操作系统层面,KV260的CPU上部署了Ubuntu Linux操作系统。Linux系统为上层应用提供了稳定的运行环境和丰富的系统服务,
包括内存管理、进程调度以及设备驱动程序等。FPGA作为一种可编程设备,其驱动和管理由Linux内核模块以及Xilinx提供的运行时库共同完成。

Xilinx运行时(XRT)是连接用户空间应用程序与FPGA加速硬件之间的关键桥梁。XRT提供了一套标准的API,
应用程序可以通过这些API来管理FPGA设备、分配和迁移数据缓冲区、加载FPGA比特流(包括用于动态重构的部分比特流)以及控制加速内核的执行。
在本系统中,我们编写的C/C++应用程序利用XRT提供的接口,实现了对FPGA上动态加载的矩阵运算模块的调用和管理,从而实现异构加速计算。

任务调度逻辑是本系统灵活性的核心体现。根据用户请求的加速计算任务类型,
该逻辑负责决策在FPGA的三个可重构分区上分别加载哪些可重构模块。
例如,若执行完整的稀疏矩阵乘法(结果为稀疏矩阵),调度逻辑可能会依次或并行地在RPs中部署稀疏矩阵解压模块、
稠密矩阵乘法模块和稀疏矩阵压缩模块。若仅进行稀疏矩阵格式转换,则会加载相应的转换模块。
这一调度过程包括了动态加载所需的部分比特流到指定RP,并通过XRT配置模块参数、启动运算、以及监控运算状态。
由于所有计算模块均支持动态重构,系统能够根据不同的稀疏矩阵格式或计算需求(例如,结果是稠密矩阵还是稀疏矩阵)灵活切换FPGA上的功能模块,
以适应各种加速场景。这种动态性是本系统区别于传统静态FPGA加速方案的关键优势。

\section{软硬件协同与系统工作流程}

本系统的核心在于软硬件的紧密协同。典型的加速计算任务流程始于用户应用程序通过XRT API发起请求。
CPU上的任务调度逻辑接收此请求,分析任务需求(如操作类型、矩阵格式、数据位置等)。
随后,调度逻辑判断当前FPGA各可重构分区上的模块是否满足需求。若不满足,则通过XRT触发动态部分重构流程,将合适的RM加载到相应的RP中。
数据准备阶段,CPU通过XRT将待处理的矩阵数据从主存传输到FPGA可访问的内存区域,或直接映射主存供FPGA的AXIMM接口访问。
一旦模块加载完毕且数据就绪,XRT便会启动FPGA上的计算内核。FPGA模块通过AXIMM接口读取数据,
执行计算(可能涉及多个RM通过AXIS接口流水线作业),并将结果写回内存。计算完成后,FPGA通过中断或轮询方式通知CPU,
CPU再通过XRT将结果数据传回用户应用程序,或通知应用程序结果已就绪。
整个过程充分利用了CPU的控制能力和FPGA的并行计算能力,并通过动态重构技术实现了高度的灵活性和资源利用率。
