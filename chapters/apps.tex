\chapter{软件系统实现与集成}

在前两章分别完成系统总体架构设计和可重构硬件模块实现后,本章将聚焦于运行在Xilinx Kria KV260平台处理器系统(PS)上的软件系统实现与集成。
这包括配置嵌入式Linux操作环境、安装和配置Xilinx Runtime (XRT)库,以及最关键的——编写主机端C++应用程序。
该应用程序负责解析用户任务、通过XRT动态管理FPGA上的可重构分区(RPs)和可重构模块(RMs)、高效地传输数据,并精确地调用和协调FPGA上的硬件加速任务。
本章旨在详述将硬件能力转化为可用、灵活的加速服务的软件实现路径。

\section{嵌入式Linux环境配置}

Kria KV260 Vision AI入门套件的处理器系统端运行嵌入式Linux。
本系统采用Xilinx官方提供的PetaLinux构建的Linux发行版,该发行版针对Zynq UltraScale+ MPSoC进行了优化,
并内建了对FPGA可编程逻辑(PL)以及动态功能交换(DFX,即动态部分重构)的支持。
环境配置的首要步骤是获取或定制包含必要内核驱动和设备树(Device Tree)配置的Linux镜像。
设备树中需正确声明FPGA管理器、可重构分区以及相关的时钟和接口资源,确保操作系统能够识别并管理PL端的硬件。
启动KV260并加载Linux系统后,会进行基础的系统配置,例如网络连接、用户管理,并安装必要的开发工具链,如GCC/G++编译器、Make等,
为后续XRT安装和主机应用程序编译提供支持。截至2025年,PetaLinux工具和Kria平台的Linux镜像已对DFX提供了成熟的支持。

\section{主机应用程序设计与架构}

主机应用程序采用C++语言编写,旨在提供一个用户友好的接口来调用FPGA实现的各种矩阵运算任务。
其核心架构围绕任务解析、FPGA资源管理(包括DPR)、数据准备与传输、以及内核执行控制这几个关键功能模块展开。
应用程序首先会初始化XRT环境,打开与目标FPGA设备的连接。随后,它会进入一个主循环或响应用户输入,
接收具体的计算任务请求,例如“执行稀疏矩阵A与稀疏矩阵B的乘法,结果以稀疏格式输出”。

任务解析模块负责将用户请求分解为一系列FPGA操作。例如,上述稀疏矩阵乘法任务可能被分解为:加载稀疏解压RM到RP1,
加载另一个稀疏解压RM(或复用RP1)到RP2(如果并行处理两个输入矩阵),加载稠密矩阵乘法RM到RP2或RP3,
以及加载稀疏压缩RM到RP3。如果RP数量有限(本设计为3个),则可能需要串行化部分操作,并将中间结果暂存到DDR内存中。
